\documentclass{report}

\usepackage[francais]{babel}
\usepackage[utf8]{inputenc}
\usepackage[T1]{fontenc}

\title{Rapport projet de programmation système 2016/2017}
\author{Groupe IN501A2\\Gautier DELACOUR\\Bérangère SUBERVIE\\Maxime VALLERON}


\begin{document}
\maketitle
\tableofcontents

\chapter{Le projet}
Ce projet a été réalisé dans le cadre de l'UE ``Programmation Système'' du semestre 5 de Licence d'Informatique de l'Université de Bordeaux. Le but était de développer des fonctionnalités destinées au développement d'un jeu de plateforme 2D à l'aide du langage de programmation C.

\section{Le jeu}
Nous avons disposé au départ de tous les fichiers nécessaires à la compilation d'un code permettant de créer un exécutable permettant de lancer le jeu. Celui-ci se présente comme une jeu de plateforme 2D, où il est possible d'ajouter ou d'enlever des objets à cette carte.

\section{Sauvegarde, chargement, et l'utilitaire Maputil}
La première partie du projet consistait à, d'une part, à permettre la sauvegarde de cartes, et leur chargement, et d'autre part, la modification de ces cartes grace à l'utilitaire Maputil.\\
Celui-ci devait permettre d'obtenir les différentes données utiles d'une carte (sa largeur, sa hauteur, son nombre d'objets), mais aussi de modifier les données de cette carte.

\section{Gestion des temporisateurs}
La deuxième partie du projet nous demander de avoir une gestion de signaux permettant la présence d'objets dynamiques tels que des bombes ou des mines.

%à compléter (Maxime)

\chapter{Le code fourni et son organisation}
\setcounter{section}{0}
\section{Arborescence}
L'arborescence du projet se trouve dans un un dossier \textbf{Projet}. Elle s'organise ainsi:
\begin{description}
\item[deps :] dossier où sont créés les fichiers de dépendances (.d) lors de la compilation.
\item[images :] dossier où sont stockées toutes les images utilisées dans le jeu.
\item[include :] dossier où se trouvent tous les fichiers en-tête (.h) définissant la signatures des fonctions dans les différents fichiers du code.
\item[lib :] dossier où se trouve la bibliothèque \textit{libgame.a} où sont codés la plupart des fonctions permettant de jouer.
\item[maps :] dossier où se trouvera les fichiers contenant les cartes sauvegardées. Il est pour l'instant inutilisé.
\item[obj :] dossier où sont créés les fichiers objet (.o) à partir des fichiers de code (.c) lors de la compilation.
\item[sons :] dossier où sont stockés tous les sons utilisés dans le jeu.
\item[src :] dossier où se trouve le fichier mapio.c (il manque cependant le code de quelques fonctions).
\item[util :] dossier où se trouve le fichier objet.txt où sont décrits les différents objets du jeu.
\end{description}

\section{Code}
Du code était déjà fourni dans des fichiers de l'arborescence de base.

\setcounter{subsection}{0}
\subsection{mapio.c}
\begin{description}
\item[map\_new :] cette fonction alloue l'espace nécessaire pour créer une carte, puis construit une carte de base.
\item[map\_save :] cette fonction devra sauvegarder une carte modifiée. Pour l'instant elle ne fait qu'afficher que cette fonction n'est pas implémentée;
\item[map\_load :] cette fonction devra charger une carte modifiée depuis le fichier \textit{saved.map} dans le dossier \textbf{maps}. Pour l'instant elle ne fait qu'afficher que cette fonction n'est pas implémentée.
\end{description}

\subsection{tempo.c}
\begin{description}
\item[get\_time :] cette fonction retourne le temps écoulé depuis le début de 2016 en microseconde.
\item[time\_init :] cette fonction est appelée au démarrage, mettra en place la gestion des signaux et initialisera les variables utiles pour le traitement. Pour l'instant, elle ne fait que retourner 0 pour montrer que le code n'est pas encore écrit.
\item[time\_set :] cette fonction arme un temporisateur d'une durée ``delay'' en microseconde et déclenchera un événement prédéfini avec sdl\_push\_event grâce à la valeur ``param''.
\end{description}

\subsection{ligame.a}
Cette bibliothèque implémente toutes les autres fonctions décrites dans les fichiers en-tête du dossier \textbf{include}.


\chapter{Gestion de cartes}
\setcounter{section}{0}
La gestion de cartes nous demandait de gérer deux principaux points:
\begin{description}
\item[La sauvegarde et le chargement:] une carte modifiée peut etre enregistrée dans un fichier saved.map, enregistré dans le dossier \textbf{maps}. Ce meme fichier peut etre utilisé pour charger cette carte éditée à la place de la carte de base.
\item[Maputil :] ce gestionnaire permet essentiellement la modification d'une carte enregistrée.
\end{description}

\section{Sauvegarde et chargement d'une carte}
Les fonctions de sauvegarde et de chargement se trouvent dans le fichier \textit{mapio.c} dans le dossier \textbf{src}.
\begin{description}
\item[Sauvegarde :] cette fonction permet de sauvegarder la largeur, la hauteur, les objets disponibles et l'emplacement des objets d'une carte dans le fichier \textit{saved.map} dans le dossier \textbf{maps}.
  \item[Chargement :] cette fonction permet de charger la carte décrite dans le fichier \textit{saved.map} dans le dossier \textbf{maps} lors du lancement du jeu.
\end{description}


\section{Maputil}
Maputil est un utilitaire de manipulation de carte. Son code source se trouve dans le répertoire \textbf{util}. Il permet de consulter des informations contenues dans un fichier \textit{saved.map} et de modifier ces informations. Il contient les fonctions suivantes:
\begin{description}
  \item[usage :] cette fonction permet de vérifier que l'utilisateur a entré une commande valide. Sinon, il arrête l'exécution.
  \item[optionsAlloc :] cette fonction alloue un tableau contenant toutes les options possibles de maputil.
  \item[optionsFree :] cette fonction libère la mémoire allouée par la fonction précédente;
  \item[getWidth :] cette fonction renvoie la largeur de la carte passée en paramètre.
  \item[getHeight :] cette fonction renvoie la hauteur de la carte passée en paramètre.
  \item[getObjects :] cette fonction renvoie le nombre d'objets d'une carte passée en paramètre.
  \item[setWidth :] cette fonction change l'ancienne largeur d'une carte par une nouvelle.
  \item[setHeight :] cette fonction change l'ancienne hauteur d'une carte par une nouvelle.
  \item[setObjects :] cette fonction change l'ancienne liste d'objets d'une carte par une nouvelle.
  \item[pruneObjects :] cette fonction supprime les objets inutilisés dans une carte de la liste d'objets de cette dernière.
  \item[traitementOption :] cette fonction teste la correspondance entre l'option demandée et les options existantes, et appelle une fonction correspondante si elle existe. Elle renvoie le nombre d'arguments utilisés.
\end{description}


\chapter{Gestion des temporisateurs}
\setcounter{section}{0}
\section{Création d'un thread démon}

\section{Implémentation simple d'une gestion de signaux}

\section{Implémentation complète de la gestion de signaux et mise en service}

\chapter{Couverture du travail demandé et son organisation}
\setcounter{section}{0}
\section{Code parfaitement implémenté}

\section{Code partiellement implémenté}

\end{document}
